\documentclass[a4paper]{article}
\usepackage[utf8]{inputenc}
\usepackage{fullpage}

\title{ Notulen vergadering groep 5 }

\begin{document}
\maketitle

\section*{Vergadering}
{\bf Datum:} woensdag 28 maart \\
{\bf Aanwezig:}\\
Albert ten Napel  \\
Niels Doekemeijer  \\
Tristan Timmermans {\it (SA)} \\
Felix Akkermans {\it (voorzitter)} \\
Thomas van Helden \\
Jan Pieter Waagmeester {\it (notulist)}\\

\subsection*{Opening}
Felix opent een vergadering. om 14:03.

\subsection*{Bespreking}
{\small Omdat het allemaal een beetje door elkaar loopt heb ik de agendapunten niet aangehouden in de notulen.}


\begin{itemize}
	\item Volgende vergadering is Jan Pieter voorzitter en Thomas notulist.
	\item Tristan: Het is handig om elke week tijdens de scrumruns even te meeten, ook als er geen vergadering geplanned is.
	\item Tristan: Er is nog geen feedback op de \textit{Test en implementationplan}. Tevredener over de \textit{Architectural Design} reports.
	\item Tristan: Doorwerkfouten worden niet meegerekend. Zitten wat rare/inconsistente dingen in, later meer.

	\item SCRUMpresentatie is niet heel ingewikkeld, laten zien wat er is en hoe het is gegaan.

	\item Enkele planningspuntjes afgesproken:
	\begin{itemize}
		\item Notulen: dag na vergadering.
		\item Agenda: dag voor de vergadering.
		\item SCRUM plan: lijst met taken klaar maken voor de eerste vergadering, estimations en toewijzingen tijdens eerste vergadering.
	\end{itemize}
	\item Tristan: Voor de SCRUM-plans: effor estimation door te bieden op taken.
	\item Tristan: testen in laatste sprint: wat bedoelen jullie daarme?
\end{itemize}


{\small Kennelijk heeft alleen Tristan inbreng in mijn hoofd... Sorry.}
\end{document}
