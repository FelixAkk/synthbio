%&latex
\documentclass{article}
\usepackage{fullpage}

\begin{document}
\title{Feedback notulen}
\author{Felix Akkermans}
\date{30 maart 2012}

\maketitle

\section{Architecture \& Design document}
\subsection{Algemeen}

\begin{itemize}
\item Docent vont het kort door de bocht, meer diepgang was gewenst
\item API had iets verder uitgewerkt moeten worden
\item Subsystem decomp was OK gedaan
\item Rood voor remarks/links is niet goed leesbaar
\end{itemize}

\subsection{Inhoudelijk}

\begin{itemize}
\item Verwarring over BioBricks/circuits: is vanwege vorige, is niet meegerekend dus
\item Wat niet duidelijk was; scheiding server/client: was niet duidelijk of de server puur voor storage/simulation was. Kwam niet helemaal duidelijk naar voren. Had eigenlijk in subsystem decomposition als eerste toegelicht moeten worden.
\item Definitions bij 1.3: goed gedaan, hoewel BioBrick/protein definitie niet helemaal klopte, maargoed.
\item References wat te kort; verwijs ook naar dingen als JSON, Tomcat, jQuery etc.
\item Vverview bij 1.5: iets teveel vraagstelling/antwoord vorm. Is gebruikelijk voor betoog, maar niet hier.
\item 2.1: Had uitgelegd moeten worden wat een servlet is. Wordt later wel duidelijk maar moest hier al gegeven zijn.
\item 2.1: ``The main advantage of using SBML is that we can easily simulate, as there are several simulators which use this format'' Welke simulators?
\item 2.2: Eigenlijk moeten verwijzen naar RAD over was .syn is, anders was het niet gedefineerd.
\item 2.2: Persistent storage staat niet in subsystem decomposition. Het is wel af te leiden dat het deel is van filesystem, maar dit is niet beschreven.
\item HTTP interface; is eigenlijk een Remote Procedure Call API, maar dit is niet herkend/uitgelegd.
\item JSON example responses: eigenlijk helemaal niet of juist voor alles moeten doen. Het is een begin aan het defineren van de API maar maakt het niet af.
\item Hardware/software mapping: goed gedaan met browser version domain definition.
\item Concurrency; verder beschrijven hoe het gaat als 2 threads naar 1 bestand schrijven.
\end{itemize}

\section{Scrum iteration plan 1}
\begin{itemize}
\item Volgende iteratie; veel aandacht op feedback: wat gehaald, wat niet, waarom, hoe fixen.
\item Geef aan hoe/waar de punten vandaan komen die je gaat doen (test\&implementation plan, moscow model etc.)
\item We hebben een iteratie van 2 weken; iedere week zou dan 8u per persoon moeten zijn, dus je kan 16u per week rekenen. Graag te zien hoe dit in iteratie terecht komt.
\end{itemize}

\end{document}


