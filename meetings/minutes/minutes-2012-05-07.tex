\documentclass[a4paper]{article}
\usepackage[utf8]{inputenc}
\usepackage{fullpage}

\title{ Notulen vergadering groep 5 }

\begin{document}
\maketitle

\section*{Vergadering}
{\bf Datum:} maandag 7 mei, 9:45-10:00 \\
{\bf Aanwezig:}\\
Albert ten Napel {\it (voorzitter)} \\
Niels Doekemeijer {\it (notulist)} \\
Tristan Timmermans {\it (SA)} \\
Felix Akkermans \\
Thomas van Helden \\
Jan Pieter Waagmeester \\

\section*{Notulen}

\subsection*{Opening}

\subsection*{Vaststelling agenda}

\subsection*{Mededelingen}

\section{SCRUM sprint 2 terugblik}
\begin{itemize}
\item JP veel gedaan en vond het leuk. Hij was het overzicht een beetje kwijt en stelt voor om meerdere meetings te houden voor duidelijkheid. Tristan vindt dit een goed punt;
\item Tristan stelt voor om ook initiatief te nemen om te vragen wat mensen doen en hoe het gaat;
\item Felix/Thomas vinden het lastig om overzicht te houden over server code;
\item Thomas had druk met congres / internet problemen. Code review was een handig punt om bij te blijven. JP vult aan dat line comments erg fijn zijn;
\item Albert heeft te laat contact gezocht met SA, daardoor had hij weinig tijd over andere punten. Ook weinig tijd voor code review over;
\item Felix vond het beter gaan dan Scrum 1, maar heeft iets te weinig tijd besteed aan code review door congres;
\item Over het algemeen beter dan Scrum 1;
\item Tristan geeft aan dat we bij ongeveer alle punten 20-30\% over de estimation heen gaan, hier moeten we op letten bij volgende iteraties;
\item Tristan is tevreden over de GUI.
\end{itemize}

\section{SCRUM Sprint 3 planning}
\begin{itemize}
\item Met hemelvaart is TU gesloten. Vrijdag staat niks ingepland;
\item Aankomende vrijdag Scrum meeting?
\item Optionele en niet afgesloten tasks worden doorgeschoven naar Sprint 3;
\item Planning wordt bepaald na de vergadering.
\end{itemize}

\section{Taakverdeling}
\begin{itemize}
\item Wordt bepaald na de vergadering.
\end{itemize}

\subsection*{W.v.t.t.k.}
\subsection*{Sluiting}

\end{document}