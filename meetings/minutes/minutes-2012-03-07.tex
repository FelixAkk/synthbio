\documentclass[a4paper]{article}
\usepackage[utf8]{inputenc}
\usepackage{fullpage}

\title{ Notulen vergadering groep 5 }

\begin{document}
\maketitle

\section{Vergadering}
{\bf Datum:} woensdag 7 maart, 16:30-16:55 \\
{\bf Aanwezig:}\\
Albert ten Napel {\it (voorzitter)} \\
Niels Doekemeijer {\it (notulist)} \\
Tristan Timmermans {\it (SA)} \\
Felix Akkermans \\
Thomas van Helden \\
Jan Pieter Waagmeester \\

\section{Notulen}

\subsection{Mededelingen}
\begin{itemize}
\renewcommand{\labelitemi}{$\bullet$}
\item Vergadering graag niet te lang, aangezien Thomas op tijd weg moet.
\end{itemize}

\subsection{Vorige notulen}
\begin{itemize}
\renewcommand{\labelitemi}{$\bullet$}
\item Notulen waren vorige keer erg beknopt.
\item Voor de volgende keer: agenda als template.
\end{itemize}

\subsection{Aktiepunten vorige keer}
\begin{itemize}
\renewcommand{\labelitemi}{$\bullet$}
\item Sequencediagram is wel gemaakt door Albert, maar was te laat voor de draft.
\item Andere akties zijn afgerond, maar soms wel na de deadline.
\end{itemize}

\subsection{Feedback draft}
\begin{itemize}
\renewcommand{\labelitemi}{$\bullet$}
\item Geschreven feedback bleek niet altijd goed te lezen, daarom heeft Tristan de belangrijke punten ook in de mail gezet en licht hij ze nog een keer toe;
\item BibTeX voor de literatuurlijst kan handig zijn, aangezien voetnoten bij veel referenties lastiger kunnen worden.
\item Introductie is goed.

\item Requirements zijn goed;
\begin{itemize}
\item Paar puntjes moeten nog wel besloten/uitgedacht worden (zoals het framework).
\item Maintainability is lastig met SCRUM. 
\end{itemize}

\item Use Cases heeft nog veel werk nodig;
\begin{itemize}
\item Bij het bestaande diagram moet nog tekst komen.
\item Zie software engineering boek voor voorbeelden.
\item Iedere requirement heeft in principe 1 Use Case (wat gebeurt er als het programma wordt geopend, wat gebeurt er bij slepen van een poort, etc).
\item De flow moet duidelijk worden uit de Use Cases.
\end{itemize}

\item Dynamic models heeft nog werk nodig;
\begin{itemize}
\item Er moeten meer models komen: bijvoorbeeld sequence diagrammen voor een aantal Use Cases.
\item Het activity diagram is goed.
\end{itemize}

\item Interface moet verder worden uitgedacht;
\begin{itemize}
\item Huidige schets is goed, maar is te oppervlakkig.
\item Alle te gebruiken schermen moeten worden toegelicht.
\item Eventueel een speciaal programma gebruiken?
\end{itemize}

\end{itemize}

\subsection{Deadlines draft}
\begin{itemize}
\renewcommand{\labelitemi}{$\bullet$}
\item Het ging niet goed met de deadlines afgelopen vrijdag; onduidelijkheid over 12u 's middags of 's avonds.
\item .docx werkt niet lekker in combinatie met Github, dus probeer uitsluitend in .tex te werken.
\item Probeer in het verslag aan te passen waar mogelijk, of eventueel in aparte .tex die kan worden ge\"importeerd.
\item Probeer op IRC te komen om kleine vragen makkelijk/snel te beantwoorden.
\end{itemize}

\subsection{Planning}
\begin{itemize}
\renewcommand{\labelitemi}{$\bullet$}
\item Verslag moet komende vrijdag af (9 maart 23:59).
\item Tristan wil vrijdag tijdens projectmiddag dingen nog wel doorkijken.
\item Werk moet tijdens projectmiddag te doen zijn.

\item Verdeling:
\begin{description}
\item[Thomas] Requirements
\item[Felix] Use Cases
\item[Jan Pieter] Interface en eindredacte
\item[Albert] Dynamic models (sequence diagrammen)
\item[Niels] Helpt Felix en Albert waar nodig
\end{description}

\item Verslag seminars moet 16 maart worden ingeleverd;
\begin{itemize}
\item Tristan zal navragen wat daar in moet komen te staan.
\item Aanpak: een iemand schrijven en de rest aanpassen of allemaal een deel schrijven (bij deze aanpak is de eindredactie erg belangrijk).
\item 1 seminar verslag mist nog in de repository.
\end{itemize}

\end{itemize}

\end{document}