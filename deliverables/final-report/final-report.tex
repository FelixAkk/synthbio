%
\documentclass[a4paper]{article}
\usepackage[utf8]{inputenc}
\usepackage{fullpage}
\usepackage{amsmath,amssymb}
\usepackage[colorlinks,linkcolor=blue]{hyperref} % use colored text in stead of ugly boxes
\usepackage[toc]{multitoc} % Nice two-column TOC

\usepackage{graphicx}
\usepackage{tabto}
\usepackage{wrapfig}

\usepackage{pdfpages}

\usepackage{pgf}
\usepackage{tikz}

%~ \usepackage{pictures/tikz-uml}

% new line after paragraph title.
\makeatletter
\renewcommand\paragraph{\@startsection{paragraph}{4}{\z@}%
  {-3.25ex\@plus -1ex \@minus -.2ex}%
  {1.5ex \@plus .2ex}%
  {\normalfont\normalsize\bfseries}}
\makeatother

% some commands to link to files.
\newcommand{\urlRAD}{https://github.com/FelixAkk/synthbio/raw/master/deliverables/rad/rad.pdf}
\newcommand{\hrefRAD}{\href{\urlRAD}{Requirements Analysis and Design @Github}}

\newcommand{\urlTestImplementation}{https://github.com/FelixAkk/synthbio/raw/master/deliverables/test-implementation/test-implementation.pdf}
\newcommand{\hrefTestImplementation}{\href{\urlTestImplementation}{Test and Implemention plan @Github}}


\title{Programming Life - Final report}

\author{Group E:\\
Felix Akkermans \\
Niels Doekemeijer \\
Thomas van Helden \\
Albert ten Napel \\
Jan Pieter Waagmeester}

\begin{document}

\maketitle
\thispagestyle{empty}
\vfill

\small{\tableofcontents}
\pagebreak

\section*{Preface} ??


\pagebreak
\begin{abstract}
\noindent This document concludes the whole \textit{Contextproject} and especially the development of our product: Zelula. This piece of software provides synthetic biologists with a visual tool to design, simulate and validate biological logic circuits.

When we compare the final product to its design, the GUI looks a lot like the sketches made in the design phase. Not only that: every must-have is recognizable in the final product.
The proposed tests are not always implemented, but a fair amount of the code is covered. We discovered that testing certains things is quite hard, so we had to rely more on acceptance testing than anticipated.

During the development we encountered a number of problems, ranging from planning and design to bugs in libraries used. One particular case of the latter was the problems we had connecting our program with an external solver. It took quite some time to debug the SBML files used in that communication.

As a team, we really think we delivered a nice product. We are also satisfied by the way the team operated.

A number of features did not fit in our schedule and will improve the program significantly, to name a few: a more thorough grouping/compound implementation, highlighting errors in the actual GUI and different workspace improvements.

\end{abstract}


\section{Introduction}
This report will detail our design and implementation of Zelula for the \textit{Contextproject}. Zelula is a piece of software for synthetic biologists with a visual tool to design, validate and simulate biological circuits.

Chapter~\ref{product-description} will contain a description of the product, every part of the finished product will be detailed. Chapter~\ref{design-implementation} contains a description of the design and implementation process, detailing how the design compares to the implementation. Chapter~\ref{problems-solutions-highlights} describes the key problems we had and explains the solutions. Chapter~\ref{team-reflection} contains a general reflection on the teamwork, followed by chapter~\ref{individual-reflection} which contains individual reflections of the team members.

Lastly appendix~\ref{scrumplans} lists our scrumplans.


%images used:
\pgfdeclareimage[width=0.45\textwidth]{gui-final}{pictures/gui_final1.png}
\pgfdeclareimage[width=0.45\textwidth]{gui-input-csv}{pictures/define-inputs-csv.png}
\pgfdeclareimage[width=0.45\textwidth]{gui-input-editor}{pictures/define-inputs-editor.png}
%~ \pgfdeclareimage[width=0.45\textwidth]{gui-final}{pictures/gui_final_spitscreen.png}

\newcommand{\screenshotScale}{1.4}


\section{Description of Zelula}
\label{product-description}
Zelula is a modelling and simulation package which enables users to simulate the protein \textit{logic} in a cell. With a provided set of BioBricks representing the basic AND and NOT gates, the user is able to design more complex circuits of interactions between the different BioBricks. The resulting circuits can be simulated and the results will be presented as a concentration/time graph.


\subsection{Circuit editor}
\begin{figure}[h!]
\centering\begin{tikzpicture}[scale=\screenshotScale]
        \pgftext[left, base]{\pgfuseimage{gui-final}}
        \draw [draw=red,very thick] (0, 2.4) -- (0, 3.7) -- (7.15, 3.7) -- (7.15,0.14) -- (1.5,0.14) -- (1.5, 2.4) -- (0,2.4);
\end{tikzpicture}
\caption{GUI mainscreen; highlighted is the workspace with the regular (draggable) gates.}
\end{figure}

\noindent Zelula enables the user to intuitively construct circuits from the basic available building blocks (see figure 1 for the main screen). Gates can be dragged from the sidebar to arbitrary positions in the working area. Connections between the gates can easily be dragged from and to the \textit{input} and \textit{output} connectors and the endpoints on gates. 

When a connection is made, no protein is selected for it by default. The user may select which protein to use. Visual feedback on protein assignment is provided through the colors of the connections.

The interface prevents the user to make some simple mistakes. For example, it's impossible to connect two inputs. If the user connects one output to two inputs, the interface makes sure the protein on those connections is the same.

Even the position of the circuits' input and output connectors can be freely chosen by dragging them using the text as an handle.

Removing gates and connection is a matter of double clicking on them.

\subsection{Compound library}
\begin{figure}[h!]
\centering\begin{tikzpicture}[scale=\screenshotScale]
        \pgftext[left, base]{\pgfuseimage{gui-final}}
        \draw [draw=red,very thick] (0,0.14) rectangle (1.5, 2.5);
\end{tikzpicture}
\caption{GUI mainscreen; highlighted is the box with the (draggable) compound gates.}
\end{figure}

\noindent Every circuit can be saved as a compound gate, which means it can easily be reused in later circuits. The saved compound gates are presented to the user in the sidebar, below the basic gates (figure 2). When a user drags a compound gate to the working area, it will expand to the original arrangement of gates, with the inner connections in place. The only thing left for the user is connecting it to the rest of its circuit.

\subsection{Validation}
When a user is satisfied with his work, he can validate the circuit. Different errors like unassigned proteins or inclomplete connections will be reported. The user will see the error message below the circuit, providing excellent overview to both the circuit and the errors to solve the problem.

\subsection{Simulation}
A valid circuit can be simulated. In order to simulate a circuit, the input must be defined as a function of time. Zelula provides two ways to accomplish this. Firstly, in an input editor the level for each input can be defined as a function of time in a visual way, secondly, a CSV text can be provided describing the transitions as a function of time.

\paragraph{Input editor}
\begin{figure}[h!]
\centering\begin{tikzpicture}[scale=\screenshotScale]
        \pgftext[left, base]{\pgfuseimage{gui-input-editor}}
\end{tikzpicture}
\caption{Dialog for defining input values per protein in client.}
\end{figure}

\noindent The input editor (figure 3) enables the user to define each input signal in a visual way. Each tick represents a number of seconds in the simulation. The user can select this number of seconds, and also the number of ticks available.

When a user clicks on a tick, it is toggled. The user may also select ranges by clicking on a certain tick an dragging the mouse to another tick while holding the mouse button.

\newpage
\paragraph{CSV input}
\begin{figure}[h!]
\centering\begin{tikzpicture}[scale=\screenshotScale]
        \pgftext[left, base]{\pgfuseimage{gui-input-csv}}
\end{tikzpicture}
\caption{Dialog for defining input values in client by CSV.}
\end{figure}

\noindent The input may also be defined by a simple CSV format (figure 4), as defined by the TA.

\subsection{Exporting}
The work done in Zelula can be shared in two different ways: the user can export the circuit to a SBML file, a common format uses in synthetic biology.

Secondly, the the graph that visualizes thee simulation output can be exported to a number of different image formats.

Of course if the user has access to the server like when it is ran locally, the user can simply save the circuit using the editor to obtain it in our application specific format (\verb=.syn=) that contains all information that was present in the application.

\subsection{Server}
The server operates in the background, serving the files and data needed for the GUI, providing persistent data storage and simulation services. The user does not interact with it in a direct way, and no server is required to be present for modelling and design. Each service of the server is provided by a Java servlet. These servlets depend on the Apache Tomcat webserver and might be distributed as a \verb|.war| archive. 



\section{Design and implementation process}
\label{design-implementation}
The first half of the project was focused on design, while the second part was all about implementation. In this chapter we will reflect on the consistency between design and implementation. We will clarify what was implemented according to the documents and what was changed or perhaps not implemented. This will done with help of a reflection on the initial GUI sketches (\ref{process-gui}) and our list of requirements (\ref{process-requirements}).
The chapter is conclude with a short section about how we finally tested our work (\ref{process-testing}).

\subsection{Client GUI}
\label{process-gui}
In our RAD document\footnote{\hrefRAD} we gave a few simple sketches of what we had in mind for the user interface of the client. The final implementation is greatly influenced by these sketches, as can be seen from the following screenshots.\\

\begin{figure}[h!]
\includegraphics[width=8cm]{pictures/gui_sketch1_marks.png} 
\includegraphics[width=8cm]{pictures/gui_sketch2_marks.png}
\caption{Sketches from RAD document: model and simulate.}
\label{fig-gui-proposed}
\qquad \\
\includegraphics[width=8cm]{pictures/gui_final1_marks.png} 
\includegraphics[width=8cm]{pictures/gui_final2_marks.png}
\caption{Screenshots of the final GUI implementation: main screen and outputs (F).}
\label{fig-gui-implemented}
\end{figure}

The matching letters in the ``before'' (figure~\ref{fig-gui-proposed}) and ``after'' (figure~\ref{fig-gui-implemented}) pictures show the same functionality. So A and B provide all the gates, D is the workspace, and so on. There have not been any great changes from what we had in mind for the user interface. We did, however, add a lot of features along the way; small features we did not think of at the beginning. For example, we colored the proteins so that they are easily distinguishable and we added general gates which serve as input or output. Another example is that used proteins cannot be chosen from during protein selection.

It is according to the SCRUM-workflow that we have added these little features. We started off with a basic working product and added them along the way.

Implementation was greatly accelerated by the use of frameworks and libraries such as jQuery, Bootstrap, HighCharts, jsPlumb and JSON-Java. The use of such frameworks of course also implies the study of their use, and solving problems or limitations encountered with them. This costed considerable time as eluded in section~\ref{problems-solutions-highlights}.

\newpage
\subsection{Requirements}
\label{process-requirements}
In the Test \& Implementation plan\footnote{\hrefTestImplementation} we set up a list of requirements according to the MoSCoW model. What follows is a copy of this list with comments on how (and if) it is implemented.

\paragraph{Must-Haves}
\begin{description}
\item[1. Connection] \textit{Client and server must be able to communicate. If there is no connection, the user should be notified.}\\
This was one of the first finished tasks, as it is essential for both client and server. The client will show an error dialog when no connection to the server can be established. While there is a connection, the user can view the connection delay (in milliseconds) in the bottom right corner of the screen.

\item[2. Available gates] \textit{The application must be able to present a list of available gates to the user. These gates can be used to model the circuit.}\\
As can be seen in figure~\ref{fig-gui-implemented}.A, the user is presented with a list of draggable (regular and compound) gates.

\item[3. Design circuit] \textit{The user must be able to design a circuit by specifying gates (using a drag-and-drop) and the relations between these gates.}
	\begin{itemize}
	\item \textbf{3.1} \textit{The application must be able to visualize a gate using a simplified image. This image should relate to the function of the gate. For example, for the AND gate, it is logical to use the distinctive AND symbol normally used in circuit design.}
	
	Gates are visualized using a simple, yet distinctive image (figure~\ref{fig-gui-implemented}.A). We simply used SVG images in the HTML to display these.

	\item \textbf{3.2} \textit{The user must be able to drag and drop gates from the list into the working area.}\\
	Gates are visualized using a simple, yet distinctive image (figure~\ref{fig-gui-implemented}.A). They can be dragged from the list and dropped into the workspace (figure~\ref{fig-gui-implemented}.D).

	\item \textbf{3.3} \textit{The user must be able freely move the gate around in the working area, but gates will snap to grid points on the working area.}\\
	After the gates are dropped into the workspace, the description of the gate disappears and the image remains. The gates can still be dragged around the screen.\\
	At first, we got the system working without a grid snapping system. This worked very well and we did not see the need to implement a grid. That is why we postponed this requirement, and decided to drop it near the end of the project.

	\item \textbf{3.4} \textit{The user must be able to draw connections between the gates in the form of wires.}\\
	In the workspace, gates get endpoints for input and output (the little green and blue dots in figure~\ref{fig-gui-implemented}.D). Wires can be dragged from the output endpoints to the input endpoints. Wires from input to input and output to output endpoints are not possible. Output endpoints can contain multiple wires, while inputs only allow one.

	\item \textbf{3.5} \textit{The user must be able to draw input and output wires for the circuit, to explicitly state which proteins will be used as input.}\\
	We have added input and output gates which serve as big input or output endpoint. Wires can be dragged from the input and dropped into the output.
\end{itemize}

\item[4. Available proteins] \textit{The application must be able to present the user with an overview of available proteins to assign to signals (visualized by the wires).}\\
Clicking Simulate $>$ Show proteins in the menu will display a table with the available proteins and matching parameters. This table is sortable and can be filtered. Each wire will also have an overlay with the selected protein. Clicking on this overlay displays a list of proteins that can still be used for selection (figure~\ref{fig-gui-implemented}.D).

\item[5. Protein specification] \textit{The user must be able to specify which protein is used for a certain signal.}\\
After opening the list of unselected proteins (figure 6.D), clicking on a protein results in this protein being specified for the connection.

\item[6. Export circuit] \textit{The application must to able to save a circuit to a .syn file.}\\
A circuit can be saved using File $>$ Save and exported to SBML using the File $>$ Export menu item.

\item[7. Import circuit] \textit{The application must be able to load an exported circuit from a .syn file.}\\
Saved files can be loaded using the File $>$ Open menu item.

\item[8. Input values specification] \textit{The user must be able to specify the input values used for the simulation of the circuit.}\\
Clicking Simulate $>$ Define inputs (or pressing F7) will open a dialog in which the input values can be defined (figure 6). These input values can also be loaded from a .csv file.

\item[9. Circuit validation] \textit{The user must to be able to validate his circuit in the application and get feedback over where there are conflicts.}\\
Pressing F8 (or clicking Simulate $>$ Validate circuit) will give a validation of the circuit. This validation is done server side, and conflicts are grounded by useful feedback.

\item[10. Circuit simulation] \textit{The application must be able to simulate a valid circuit and present the output values to the user.}\\
Pressing F9 (or clicking Simulate $>$ Run solver) will simulate on the server and present the user with a plot of the output values (figure \ref{fig-gui-implemented}). The user can zoom in/out and specify which proteins are visible in the plot (only the input and output values are shown by default). Because we use a simulation library which supports multiple solving methods, we give the user the options to select a method.
\end{description}

\paragraph{Should-Haves}
\begin{description}
\item[11. Re-use circuits] \textit{The application can import pre-defined circuits as extra gates. This is not a necessity, but would be a great addition to the program (and will ease building circuits). Among others, protein specification, importing and exporting will be more difficult to implement.}\\
This was a requirement we really wanted to add, but decided to implement a simpler version due to time restrictions. We wanted to make it so that compound gates were visualized the same as regular gates, but decided this would be too much work. We can now mark circuits as compound gates at the time of saving. These gates will then show up in the list of compound gates and can be dragged into the workspace. When dropped, the old circuit will show up (minus the input/output signals).
\end{description}

\pagebreak
\subsection{Testing and test coverage}
\label{process-testing}

Since the development of this project was to be test-driven, we did put some effort in writing a test and implementation plan\footnote{\hrefTestImplementation}. This section reflects on the use of that document, and on the amount of testing we actually did.

\paragraph{Unit testing}
Unit testing was a major part of our test plan. Having written a test plan implies the actual use of it, however, we did not really use the plan in deciding what to test. Tests were mainly written together with the code, not necessarily before implementing it.

Unit testing client side JavaScript code was planned, but not implemented in the same depth as intended. GUI interaction was not tested in any automatized way, mainly because we did not know how to do it. It might have been possible to invest more time in it, but since the structure of the interface did change a lot it would have been really cumbersome. Testing these aspects of the program was done through acceptance tests, with a detailed description below.

Judging from the past, we lacked the experience to write a decent doable test plan. We made some bold and broad plans but both too precise and too vague. More emphasis on the testing during the development could have fixed it, but was omitted.

\paragraph{Test coverage}
% reflection on real coverage, as well as branch coverage for js an java.
Testing for coverage was not really covered in our testing plan, however, during the selection of tasks for iteration~4 we decided to explicitly check and improve coverage. We used Cobertura to report on Java coverage, and JSCoverage to report on JavaScript coverage.

As can be seen in figure~\ref{fig-java-coverage}, branch coverage figures are moderately high, but not exceptional. Apart from that, coverage on issues not easily recognizable by coverage reports like these is not always very good. For example, a more detailed plan on the things a test for a solver should cover could have exposed the simulator problems earlier in the process\footnote{For a detailed explanation on this problem, refer to section~\ref{simulator-problems}}.
\begin{figure}[h!]
	\includegraphics[width=\textwidth]{pictures/java-coverage-packages.png}
	\includegraphics[width=\textwidth]{pictures/java-coverage-models.png}
	\caption{Overview of Java package coverage and the coverage of a selected package.}
	\label{fig-java-coverage}
\end{figure}
\begin{figure}[h!]
	\centering\includegraphics[width=.7\textwidth]{../../screenshots/2012-06-17-jscoverage-report-small.png}
	\caption{Overview of JavaScript coverage.}
	\label{fig-javascript-coverage}
\end{figure}


\paragraph{Acceptance testing}
 The acceptance tests
are test if the system meets the requirements made, and how the system performs on pseudo-requirements.
In particular we test these using the user interface to see if the system can perform the required tasks without
problems. With these test we also manually test the user interface to see if there any display artifacts or
bugs. Because these acceptance test call all the functionality throughout the whole system from the interface,
they also serve as an important part of the integration testing.

The acceptance testing included the running of several specified scenarios, and a grading on several usability aspects to measure how well the application performs there.

In the end these tests were write quite late, and we only came to performing these test in the last iteration (SCRUM iteration 5). We had several people perform these tests, some developers, and some other computer science students, as these fall within the category of target users. Despite this we these test runs revealed a lot of bugs and gave us valuable feedback that influenced us in making changed in the application during the last iteration. Some structured reports can be found attached at the end of this document.


\pagebreak
>>>>>>> origin/master
\section{Key problems and solutions - highlights}
\label{problems-solutions-highlights}

During this project we have encountered some problems. In this section we will have a look at some of the highlights of these problems and how we fixed them.

\paragraph{Planning and design}
Since, for some of us, this was the first time working with SCRUM, it took some getting used to. At first, we had to get used to working towards a result in two weeks and estimating what we could do. This quickly became easier as the project progressed. 
What we really noticed is that our design was not complete. We hadn't thought of the complete structure and failed to set some standards of coding. This led to less readable code, which is annoying when reviewing each others code, and no clear idea which function was implemented where. Especially later on, some functions turned out to be double implemented or unused. This point is also discussed in the paragraph about restructuring.

\paragraph{Setup}
A common issue using server-client structure is the actual connection between the two. In the beginning we had some trouble with this but this was mainly caused by out of sync code and trouble with Apache Tomcat. We decided on a standard version of Tomcat and wrote a setup page on the wiki\footnote{\url{https://github.com/FelixAkk/synthbio/wiki/Tomcat-installation}}. Afterwards, the problem was solved.

\paragraph{QUnit}
Our program was developed test driven. This makes for very neat code, but it also comes with some problems. We mainly encountered problems with QUnit, a JQuery testing framework similar to Java's JUnit. Although the testing was done properly, the feedback on the results was not very helpful. Eventually we managed to tackle this problem with some extra manual testing. We chose to do some extra manual testing instead of switching testing framework because of all the work we already put into it. Also, QUnit works really easy with JQuery code and the code is readable. 

\paragraph{Drag and drop}
Another problem client side was the dragging and dropping of gates and connecting them with wires. We had anticipated this as being tough and found a solution in jsPlumb, a solution with its downsides. The framework jsPlumb makes it easy to create connectors and draw wires between gates, but it didn't fully meet our needs so we had to customize it, leading to quite some hours of extra work.

\paragraph{Dropdown menus}
At one point, we figured we wanted to select our proteins from dropdown menus on our wires. Regular HTML dropdown menus are not an issue, however combining this with jsPlumb and bootstrap proved to be a little more difficult. This took more time than we had anticipated but was eventually solved by fully generating the menus every time we needed it. When opening a menu we just adjust its size, to simulate its dropdown effect.

\paragraph{Communication and github merging}
Throughout this project we worked with Git and GitHub to share our code. We also used the ticket mechanism of GitHub to keep track of what had to be done. Git is exceptionally good at merging code but cannot always prevent conflicts. If there is too little communication between two people who are working on the same code, merging can become difficult. However, with some reverts and recommitting these problems were resolved.

\paragraph{Restructuring}
We had a lot of code clutched together. We had that idea ourselves and the SIG evaluation gave the same result. So towards the end of our project, we did some major restructuring. We shifted some functions around, deleted unused functions, changed some names and did some more testing and commenting. On the client side we put every function into files named exactly after the part of the program they run in. This made it a lot easier to look for functions when you wanted to adjust code, or notice an error in some part of the program. 

\paragraph{Solver issues}
During the second to last iteration we discovered a problem with the connection to the solver which was implemented. It didn't show correct results for the transcription and translation reactions, because degradation didn't work. After discussing things with Alexey, Jan Pieter decided to write a solver on its own, which initially took about 12 hours of work. Cleaning up the code and some other adjustments took another 8 hours.
During the development of our own solver, Albert worked together with Alexey to fix the problems in the first approach. The result of this is the availability of two solvers in our GUI.

\paragraph{Graph rasterisation, servlet issues}
The graph library used in our GUI provides a way to export the graph to different image formats. This exporting requires a special server which is provided by the author of the library, but we wanted to be independent of that, so we decided to provide the rasterisation service by our own server code.
At first, it seemed very straight-forward, but quite some time was absorbed by the confusion about Tomcat's class-path. It turns out Tomcat only looks in \verb|WEB-INF/lib/*| and not in it's subdirectories, resulting in a huge stack of \verb|NoClassDefFoundErrors| to fix.

\paragraph{Migrating GUI elements from modals to a workspace tab}
The GUI was initially implemented with several key GUI elements such as the output, input and validation results in modals. Although this was not according to the final designs, and has several drawbacks on usability when working on a circuit alongside, this was far easier to implement than in an element that was always present in the workspace. This element had high requirements on flexibility (being able to hide, show again without overlapping the workspace, resize to meet size requirements, go fullscreen in a usable way etc.), which was not provided by any ready made framework or library. This meant implement a lot of workspace management and crafting on these tabs. As work progressed on this it turned out to be a more and more complex task, going far over the estimated time. At some point we had to decide which workspace we'd write the report about and deliver at the last scrum. Although work was nearly completed, there was just not enough time and we left this feature out.
\documentclass[a4paper]{article}
\usepackage[utf8]{inputenc}
\usepackage{fullpage}
\usepackage{amsmath,amssymb}
\usepackage[colorlinks,linkcolor=blue]{hyperref} % use colored text in stead of ugly boxes
\usepackage[toc]{multitoc} % Nice two-column TOC
\usepackage{graphicx}
\usepackage{wrapfig}
\usepackage{tabto}
\usepackage[font=small]{caption}

\usepackage{pdfpages}
%~ TikZ is not in use in this document
%~ \usepackage{pgf}
%~ \usepackage{tikz}
%~ \usepackage{pictures/tikz-uml}

% new line after paragraph title.
\makeatletter
\renewcommand\paragraph{\@startsection{paragraph}{4}{\z@}%
  {-3.25ex\@plus -1ex \@minus -.2ex}%
  {1.5ex \@plus .2ex}%
  {\normalfont\normalsize\bfseries}}
\makeatother


\begin{document}

\section*{Simulator problems}
In the second to last iteration we discovered that our simulator didn't degrade anything (see figure 1), this was discovered so late because non of the tests targeted degradation, we only tested if the reaction speed changed. Either the solver library (SBMLsimulator) was wrong or the SBML format used was. After some testing and help from Alexey, the problem found was indeed the SBML format.\\

\begin{figure}[h!]
	\centering\includegraphics[scale=0.35]{../../screenshots/2012-05-31-not-output.png}
	\caption{Wrong output, B doesn't degrade.}
\end{figure}

\noindent The SBML format that is used for the simulation requires a list of species and a list of reactions. These reactions each have a set of reactants, products and modifiers:
\begin{itemize}
	\item reactants are the fuel of the reactions, the reactants decrease as the reaction occurs,
	\item products are what the reaction produces in the end, the products increase as the reaction occurs,
	\item modifiers can speed up or slow down the reaction, but their concentration doesn't change in the reaction.
\end{itemize}
In our simulation we have two reactions: transcription and translation. The transcription reaction produces mRNA (that encodes for a specific protein) from a gene sequence with help from transcription factors. The translation reaction produces the protein from the mRNA. On top of this degradation takes place: the proteins and mRNA degrade over time.\\
\\
In our SBML file the degradation didn't have a separate reaction, which caused a problem given that the degradation reaction required a different set of reactants, products and modifiers from the transcription and translation reactions.\\
The transcription reaction produces something from a gene sequence. The degradation reactions need to degrade to something, but it's not producing something from a gene sequence, so putting the degradation in the same reaction as the transcription (or translation) reaction would be wrong since they have different reactants and different products.\\

\begin{wrapfigure}{l}{8cm}
	\centering\includegraphics[scale=0.5]{../../screenshots/2012-06-14-not-output.png}
	\caption{Correct output, B degrades.}
\end{wrapfigure}

\noindent To fix this (see figure 2) we split the transcription and translation reactions such that the degradations got their own reactions, this allowed us to specify different reactants, products and modifiers for the degradation.
\end{document}

\subsection{Late external requirements}
\label{late-requirements}
During the project we received several new ro changed requirements from the project leaders/TA's. Because these were late this often mean we had to suddenly change our planning to accommodate for the unexpected work. Some of these requirements were:
\begin{enumerate}
\item The capability of the user to specify the input values for the simulator in a CSV\footnote{Comma seperated value. A simple, generic scheme for structuring information for transport. The format is outlined in Handout 2, available on the Blackboard course.} format, somewhere in the GUI. It was later communicated that the format was amended due to limitations of the initial format, and required that the application was adjusted accordingly.
\item Some requirements and guidelines on this final document were communicated recently.
\end{enumerate}

\noindent Although this sometimes brought up some inconveniences in the development process, we then realized ourselves that this is actually quite typical of a real world development scenario, and accepted that this is an aspect where flexibility from our part is rightly  to be expected. In this line we quickly managed to adjust our system to these new requirements.

In the end this was probably a good exercise, and something we coped with properly.


\pagebreak

\section{Reflection on the teamwork}
\label{team-reflection}
The teamwork during this project grew better as time passed. We got more used to each other and everyone became a little more enthusiastic during the implementation phase. The first phase, the design phase in which we wrote a lot of documents, passed without any big problems. We had a meeting every week in which we divided tasks and had little brainstorms. These brainstorms brought up a lot of ideas and we were eager to start implementing.

The second phase, implementation, started off very well with good results after the first iteration. Again we had weekly meetings to divide work and to discuss ideas, but a lot of the communication was now done casually. We came together more often while implementing, so the threshold to ask questions and throw out ideas lowered. The rate at which code emerged was pretty high.

This high activity occasionally resulted in work being done twice, but these miscommunications were resolved well. There were no other notable problems, other than perhaps some implementation problems. Such implementation problems were handled by having multiple people look at the problem and discussing possible fixes.

Every project member is proud of the delivered product and is satisfied with how the project elapsed. The next chapter will give a personal reflection of each team member on the project.

\section{Individual reflection on the project} %max 1 A4 per teammember
\label{individual-reflection}
In this section each team member gives a summary of their experience and analysis of the project.
\subsection{Felix Akkermans}
The project was the longest spanning one I've participated in thus far. Especially the synthetic biology/programming life direction really appealed to me. I had very high hopes, and am very happy that with this project we managed to realize something great.

During the start the project took a very different direction than expected, which caused the necessary confusion during the documentation phase when it came definitions and work that was to be expected. At the same time we had a course in project management skills, which I think certainly helped us cope with this, although not ideally as fast and flexible as we could have. But then again this was the first time we were actively and consciously trying to apply these skills. At several occasions I really noticed this made a difference in insight and outcome.

The documentation phase went quite smooth, with an increase of quality in our work as we managed feedback from the tutors.

When we started on the implementation we paid special attention to the novel project management method (SCRUM) we were going to use. I think we coped with this very well, and got up to speed with it's iterative rhythm very well. From the first iteration we have a working presentable product, and managed to stick to this pattern throughout the project. During each iteration we had our share of teamwork and implementation problems. The implementation problems we coped with reasonable I think. Be it slow and gradual, in general the teamwork and improved. I felt that for some the dedication to the project declined quite at some point, but in the end everyone delivered a good contribution to the project.

My work has mostly focussed on the GUI. I feel that my work and criticism was appreciated here and this also made it a great job to work on. I felt that I made a useful and good contribution with directing the GUI, but I wish I had more time to spend at reviewing work of others and helping in their fields. This often left me with a feeling of a lack of overview of the project. 
\subsection{Niels Doekemeijer}
Looking back at this project, I am quite pleased with the overall teamwork and final result. I feel like we had a bit of a rough start, having very few meetings in the first part of the project. This resulted in somewhat lower grades than expected for the first few documents. Writing reports is not my favourite part of projects, but it has to be done and I did my share.

After starting implementation of the product, I feel like the group came more at ease (having more weekly meetings definitely helped, too). I had never worked with SCRUM before, but I can see why it's accepted as a development method as it definitely helps to always have a working product. Demonstrations after each iteration helped us confirm that we were well on our way. We put a lot more work in the implementation than in the reports, but I think both were of very decent level.

I started taking more initiative in the implementation phase and tried to always have an overview over what was happening and what needed to be done. My opinion is that every group member had a phase in which they did a lot of work and a phase in which they did less than usual. I think the overall activity and devotion to the project was good. This has led to a decent product which has exceeded my initial expectations.

This high activity and devotion sometimes lead to "code-clashes" as more people accidentally worked on the same bit of code. I personally think these little conflicts are no big deal and I rather have work being done twice than work not being done at all. It did not affect our team work and I think we settled these few miss-communications well.

\subsection{Thomas van Helden}
I am quite amazed about the product we managed to put together. I'm certainly glad we made it with a nice JS GUI. Previous projects were all made in Java leading to horrific GUIs. This project is also the most complex product I've ever made. So in several ways, this was the most challenging IT project I've worked on. Once again, I've learned that a good preparation works a million times better than just starting to code. 

My reflection on the teamwork can be divided into three parts: A general part, feedback on my work and communication about the product.
In general I was very pleased with the team. We all started every sprint with a meeting. We discussed the issues in the previous sprint and we divided the work. Although some did more than other, we all had an understanding mindset towards each other.

I believe the biggest part was the feedback on me. I personally feel as if I really didn't do the job I could normally do. I was occupied with a lot of things, personal and professional, and so I was unable to do my job properly. This was discussed in the team and I got some really good feedback on how to improve my attitude towards the team. 

Communication, in what I've learned so far, is always an issue. Although we did a good job, there was still some miscommunication leading to extra work and stress. It happened a few times that people were working on the same thing without the other one knowing. This resulted in two people doing double work, wasting hours of time. This was no real disaster but it was a shame. Apart from these few points there was nothing to mention with regard to our teamwork.

\subsection{Albert ten Napel}
I think the project has been very successful, the resulting product is a beautiful, easy to use application that has quite a few useful functions. The teamwork itself was pretty good, we had a meeting each scrum iteration planning on what we would do that iteration. This way I knew what I should do and what was expected of me. In previous projects I was often not sure on what I was to do and the planning was really weak. We also used GitHub, which allowed us to create issues detailing things like optional tasks, bugs and unassigned work. When I was finished with a task and I didn't have anything to do, I could check the issues and pick a new task.

Communication was mostly good, if somebody had a problem he could tell it to the group and we would all think about it. There was some miscommunication regarding tasks that two people were doing at the same time without the other knowing. But in the end this was solved fairly quickly and wasn't that big of a problem.

During the project I learn to better manage things, for example using tools like ant. I also learned how to set up a server-side system with Tomcat and Java Servlets and how to manage libraries in Java. Furthermore we used git instead of svn (which was used in earlier projects), I learned a lot about how to use git.

Test driven design wasn't something that I liked a lot. In theory I like the idea but in practice I find writing all of the tests upfront more of a burden and a lot of time was spent actually fixing tests instead of fixing code.

I didn't do a lot of client side, JavaScript things, because early on I wasn't very interested in that part of the application. Later on I didn't feel confident that I could help with the JavaScript stuff because I wasn't very familiar with the client side. I think I should have done a bit more, since I would've learned some things and with a next project I would like to look more into the GUI side of things.

\subsection{Jan Pieter Waagmeester}
The first thing I would like to note is that I am surprised by the final product we managed to deliver, it exceeds my expectation. It looks good, works well, has clean code and all the must-haves we defined. Overall, I am also satisfied with the way the team worked together. There were some frustrations now and then, but always room for feedback. 

I will reflect in more detail on two areas: the technical development and the team collaboration.

\paragraph{Development \& tool use}
% Development process, TDD, tool use.
It was quite some time ago since I did collaborative Java development on larger scale. So I did not have strong opinions about the tool chain to be used, however, during the Software testing class I got my share of Eclipse frustration, so glad to leave that for a while.

I'm used to the command line and enjoyed learning to use new tools like \verb|tomcat|, \verb|ant|, \verb|git|. Tomcat is a very complicated way to have simple server functionality. Since we only needed exactly that, it might have been more efficient to select something more lightweight, however, lots of documentation is available. Ant is a powerful way of to automate boring tasks and made me smile from time to time. Git, especially in combination with GitHub beats every other SCM toolchain I have used before. Easy of branching makes it useful, local commits make code review by peers way easier, but like any other SCM depends on how it's used.

The actual development was often quite straightforward. No very complicated algorithms, mostly pushing simple data structures around. That said, testing things is not always easy. For example, testing JSON responses is not a trivial task if the testing tools have no notion about its structure. I think we could have put more focus on exactly what to test, not only on line or branch coverage.

I really did like the test driven design approach, or more precisely, the fact that tests existed. Some refactoring made the importance of tests very clear: the tested parts worked like before because problems were identified by the tests, but the untested functionality lagged behind.

\paragraph{Team collaboration}
% balance of work, Initiative. SCM frustration
Balancing the work during projects is an always an issue. Some people have less time available, some like to invest more time. Some are perfectionists, others do not care about brace placement. I think being one of the more active people does not always justify complaints about others contributing less. Some people could have displayed more initiative from time to time.

Working together in a source code management system can be a source of frustrations: braking \textit{the make}, little care when committing unintentional changes or merges. I sometimes found it hard to patiently correct them and explain the how and why.

\section{Future implementations}
Throughout this project we implemented a lot of features, but we were not able to implement all our ideas. In this section we will explain some features which could be implemented in the future. Some of these are would-likes we thought of during the design phase, others were ideas we though up during implementation. We tried to make our application in such a way that there is a good base for these ideas.

\subsection{Acceptance testing results}
From our acceptance tests we got a lot of feedback. One of the main goals is to improve the things mentioned there, because those are the bugs and inconsistancies that users noticed. One example of this, is that saving an unnamed file with a certain name, does not change the circuits name in the description. This was mentioned in the acceptance test as being a feature, but it didn't work. We already fixed some issues we noticed during our earlier acceptance tests. In the appendix you can find more about the results of these acceptance tests.

\subsection{Groupings}
One of the ideas we had was the concept of groupings. We would use groupings to specify the structure of a circuit. This would be easily converted to JSON and would help to load files to the workspace, directly displaying every circuit with proper spacing and placement. This was not implemented mainly because of time deficiency.

\subsection{Protein concentration}
Another idea was the ability to set the protein concentrations. Within a nice interface, one could easily specify which proteins would have which concentration, and perhaps even at which time the protein would start. This idea was also encouraged by the TA's but unfortunately we didn't have enough time.

\subsection{Highlighting validation error}
When a circuit validates it says in green: "Circuit validates!". When there are some errors, it displays in text which errors are present. What would be nice if the element causing the error would light up in the workspace. So for instance, a wire has no protein specified, then that wire would light up.
If a gate has an open input slot, that gate would light up, or maybe that input slot.

\subsection{The green dot}
Currently, there is a bug in the GUI. Sometimes the end point of the last output wire, connected to the output panel, is displayed at the top left corner of the workspace. This bug only occurs once in a while and after hovering over the wire or shifting gates or panels, it disappears. We did not fix this issue because it is probably an issue within the jsPlumb library we use and we did not have time to fix such a small bug.

\subsection{Compound and normal gate distinction}
Compound gate are now saved as regular gates, but also separately in a compound folder. So to change a compound gate you have to change the regular circuit and resave it as a compound gate. It would nice to have this separated, so compound gates could be altered by itself.
We chose not to do this because of time restrictions. We do have the foundation for this to work, so implementing it will be rather easy.

\subsection{Compound gate display}
This was one of the major things we would like to add. If you drag and drop a compound gate, it shows the circuit it is buildup of. Our original plan was to display these gates as a small gate, on which you could zoom in or double click to see the entire circuit. This was rather complex, and we did not have enough time to implement this unfortunately.

\subsection{Input and output fields display open wires}
What would be nice for our input and output panel, is to have connection slots. So per available connection there would be an open connection slot generated on the input and output panel. This would make it more visible when a connection has not been made yet. Also it would generate standard points from where they wires would start. This would prevent the weird shifting of wires, which sometimes happens when resizing the window or dragging gates.
We saw this point of improvement as too small to actually implement it, with regard to our available time.

\subsection{Zoom workspace}
Right now we use the browsers standard zoom to zoom in and out. But this also changes the menu and side panel size. This is undesirable, so what we would like is to have a zoom which only affects the workspace. Perhaps in combination with the groupings, you could zoom out to groupings level. In combination with the compound gates you would zoom in on them to see their inner circuit. This was also too big to implement.




% what follows are appendices.
\appendix

\section{Lightweight SCRUM Plans}
\label{scrumplans}
The following pages will include the relevant parts of our lightweight scrumplan for each iteration. The colored links and numbers with a hashtag refer to GitHub issues on \url{https://github.com/FelixAkk/synthbio/issues}. For each iteration, a milestone is available providing a nice overview of the issues for each iteration.

Comparing actual time used to the original estimation shows something about how accurate the estimates were. A comparison can be found in the following table. The comparison shows that we managed to estimate the work quite precisely, however, uncompleted tasks which were moved to the next iteration are not shown. The last iteration shows a major difference because we could not postpone tasks to another iteration anymore so we invested more time to finish everything.
\begin{table}[h!]

\centering\begin{tabular}{lrrr}
 & \multicolumn{3}{c}{Effort} \\
 \cline{2-4}
Iteration & Estimated & Actual & Comparison \\
\hline
\#1 & 65 & 68 & 4,62 \% \\
\#2 & 92 & 87 & -5,43 \% \\
\#3 & 93 & 98 & 5,38 \% \\
\#4 & 73 & 72 & -1,37 \% \\
\#5 & 77 & 112 & 45,4 \% \\\hline
\end{tabular}
\caption{List of estimations, actual times and a comparison for each iteration}
\end{table}

\includepdf[
	landscape, frame, nup=1x2, pages=2-3,
	addtotoc={2 , subsection , 1 , Scrumplan 1 , scrum-1}
]{../scrumplan-1/scrumplan-1.pdf}

\includepdf[
	landscape, frame, nup=1x2, pages=2-3,
	addtotoc={2 , subsection , 1 , Scrumplan 2 , scrum-2}
]{../scrumplan-2/scrumplan-2.pdf}

\includepdf[
	landscape, frame, nup=1x2, pages=2-3,
	addtotoc={2 , subsection , 1 , Scrumplan 3 , scrum-3}
]{../scrumplan-3/scrumplan-3.pdf}

\includepdf[
	landscape, frame, nup=1x2, pages=2-3,
	addtotoc={2 , subsection , 1 , Scrumplan 4 , scrum-4}
]{../scrumplan-4/scrumplan-4.pdf}

\includepdf[
	landscape, frame, nup=1x2, pages=2-3,
	addtotoc={2 , subsection , 1 , Scrumplan 5 , scrum-5}
]{../scrumplan-5/scrumplan-5.pdf}

\includepdf[
	landscape, frame, nup=1x2, pages=1-2,
	addtotoc={2 , section , 1 , Acceptance testing , acceptance-testing}
]{../acceptance_testing/acceptance_testing_results.pdf}
\end{document}

 product will be detailed. Chapter 3 contains a description of the design 

