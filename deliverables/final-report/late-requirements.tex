\subsection{Late external requirements}
\label{late-requirements}
During the project we received several new or changed requirements from the project leaders/TA's. Because these were late this often mean we had to suddenly change our planning to accommodate for the unexpected work. Some of these requirements were:
\begin{enumerate}
\item The capability of the user to specify the input values for the simulator in a CSV\footnote{Comma seperated value. A simple, generic scheme for structuring information for transport. The format is outlined in Handout 2, available on the Blackboard course.} format, somewhere in the GUI. It was later communicated that the format was amended due to limitations of the initial format, and required that the application was adjusted accordingly.
\item Some requirements and guidelines on this final document were communicated recently.
\end{enumerate}

\noindent Although this sometimes brought up some inconveniences in the development process, we then realized that this is actually quite typical of a real world development scenario, and accepted that this is an aspect where flexibility from our part is rightly  to be expected. In this line we quickly managed to adjust our system to these new requirements.

In the end this was probably a good exercise, and something we coped with properly.
