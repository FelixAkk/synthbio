\documentclass[a4paper]{article}
\usepackage[utf8]{inputenc}
\usepackage{fullpage}
\usepackage{amsmath,amssymb}
\usepackage{tabularx,multirow}
\usepackage[colorlinks,linkcolor=blue]{hyperref} % use colored text in stead of ugly boxes
%~ \usepackage[toc]{multitoc} % Nice two-column TOC


% new line after paragraph title.
\makeatletter
\renewcommand\paragraph{\@startsection{paragraph}{4}{\z@}%
  {-3.25ex\@plus -1ex \@minus -.2ex}%
  {1.5ex \@plus .2ex}%
  {\normalfont\normalsize\bfseries}}
\makeatother

\title{Programming Life - Scrum plan iteration 1}

\author{Group 5/E:\\
Felix Akkermans \\
Niels Doekemeijer \\
Thomas van Helden \\
Albert ten Napel \\
Jan Pieter Waagmeester}

\begin{document}
\maketitle

%~ • The selection of a set of features (important features first).
%~ • A List of the tasks for each feature.
%~ • The assignment of students to tasks.
%~ • An estimation of the effort per task.
%~ • The actual effort per task (after the iteration is done).
%~ • Short reflection on the main problems and adjustments of the iteration planning
\section{Introduction}

\section{Selection and assignment of tasks}
Since this is the first iteration, it is hard to predict how things work out. Lots of things are to be sorted out and we have to become familiar with the tools to be used.

The time available is two afternoons in two weeks by five people. That's about sixteen hours per person, however, some time should be reserved for the scheduled meeting on wednesday. So the total effort available is about 40 man hours.

\begin{table}[h!]
	

\begin{tabularx}{\textwidth}{p{9cm} l cc}
	& & \multicolumn{2}{c}{Effort} \\	
Task & Developer & estimated & actual \\
\hline
Server: Tomcat server that handles HTTP requests &
 & 8 & \\
 
Server: Basic MVC framework, so that future components can easily be integrated &
 & 8 & \\
 
Client: Create an index.html with a basic GUI &
 & 8 & \\
 
Server: Serve available proteins and gates &
 & 8 & \\
 
Client: Show the connection state; &
 & 4 & \\
 
Client: Show available gates &
 & 8 & \\ \hline

\textbf{Total} &
 & 44 & 
\end{tabularx}
\end{table}
%~ \section{Reflection on this iteration}

\end{document}
