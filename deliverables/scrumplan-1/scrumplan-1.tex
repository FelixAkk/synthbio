\documentclass[a4paper]{article}

% scrumplan-common.tex

\usepackage[utf8]{inputenc}
\usepackage{fullpage}
\usepackage{amsmath,amssymb}
\usepackage{tabularx,multirow}
\usepackage[colorlinks,linkcolor=blue]{hyperref} % use colored text in stead of ugly boxes
%~ \usepackage[toc]{multitoc} % Nice two-column TOC

\usepackage{ifthen}

% new line after paragraph title.
\makeatletter
\renewcommand\paragraph{\@startsection{paragraph}{4}{\z@}%
  {-3.25ex\@plus -1ex \@minus -.2ex}%
  {1.5ex \@plus .2ex}%
  {\normalfont\normalsize\bfseries}}
\makeatother


\setlength{\extrarowheight}{3pt}
\newcommand{\githubissue}[1]{\href{https://github.com/FelixAkk/synthbio/issues/#1}{\##1}}
\newcommand{\githubmilestone}[1]{\url{https://github.com/FelixAkk/synthbio/issues?milestone=#1}}
\newcommand{\thickhline}{\noalign{\hrule height 0.8pt}}

\newcommand{\tasktableheading}{
	\multicolumn{3}{c}{} & \multicolumn{2}{c}{Effort} \\
	& \textbf{Task} & Developer & estimated & actual \\\hline
}

\newcommand{\subheading}[1]{
 & \multicolumn{2}{l}{\bf #1} \\ \hline
}

\newcommand{\subtotal}[2]{
	\hline
	\multicolumn{3}{r}{\textbf{Total}}
	& #1
	& #2 \\[6mm]
}

\newcommand{\grandtotal}[2]{
	\thickhline
	\multicolumn{3}{r}{\textbf{Grand Total}}
		&
	#1 &
	#2 \\
}

%task command to 
\newcommand{\task}[5]{
	\ifthenelse{\equal{#1}{} \or \equal{#10}{0}}{-}{\githubissue{#1}} &
	#2 & % description
	#3 & % developer
	#4 & % estimated effort
	#5   % actual effort
	\\
}



\title{Programming Life - SCRUM plan \#1}

\author{Group 5/E:\\
Felix Akkermans \\
Niels Doekemeijer \\
Thomas van Helden \\
Albert ten Napel \\
Jan Pieter Waagmeester}

\begin{document}
\maketitle

%~ • The selection of a set of features (important features first).
%~ • A List of the tasks for each feature.
%~ • The assignment of students to tasks.
%~ • An estimation of the effort per task.
%~ • The actual effort per task (after the iteration is done).
%~ • Short reflection on the main problems and adjustments of the iteration planning
\section{Introduction}
In this SCRUM plan we define the tasks we want to complete in this sprint. Every task is assigned to one or two developers and an estimate is provided for the effort required.

For each SCRUM run a milestone is created at GitHub, with issues for the tasks selected. The issues for this milestone can be found on: \githubmilestone{8}.

\section{Selection and assignment of tasks}
Because this is the first iteration, it is hard to predict how things work out. Lots of things are to be sorted out and we have to become familiar with the tools to be used.

\paragraph{Available time}
The time available is four afternoons in two weeks by five people. That's about sixteen hours per person, however, some time should be reserved for the scheduled meeting on wednesday. So the total effort available is about 75 man hours. When writing this document, we made an error in this calculation resulting in a total effort available of 40 man hours, so during the estimation of the tasks we tried to fit all the work in those 40 hours. This might be the reason why there was some underestimation for the tasks in this iteration.

A list of tasks, assignments and effort estimations is included in a table.


\paragraph{Presentation}
Results for this iteration will be presented thursday april 5, 16:30 in the library (project room Australia).


\begin{center}
\begin{tabularx}{\textwidth}{r p{8cm} | l | cc}
\tasktableheading

\task{2}
	{Server: Tomcat server that handles HTTP requests {\sc Hello World} and class scaffolding}
	{Niels/Albert}{2*10}{17}

\task{3}
	{Server/Client: Design and document JSON format for circuit (\verb|.syn|) and protein list.}
	{Jan Pieter}{4}{3}

\task{4}
	{Client: JavaScript scaffolding}
	{Thomas}
	{8}{10}
 
\task{5}
	{Client: Create basic GUI}
	{Felix}
	{8}{}
 
\task{6}
	{Server/client: create and display connection state by ping/pong heartbeat}
	{Niels/Albert}
	{2}{3}
 
\task{7}
	{Client: Show available basic gates}
	{Jan Pieter}
	{3}{1 + ??Felix}

\subtotal{45}{??}
 
\subheading{
	Optional tasks\footnote{Things from next iterations that could be done if sufficient time is available}
}

\task{8}
	{Server: Parse and serve list of Proteins\footnote{Only after list is supplied by Alexey}}
	{Jan Pieter\footnote{Original plan was \textit{Niels/Albert}, but Jan Pieter finished his tasks...}}
	{4}{16}

\task{11}
	{Cient: Display list of Proteins}
	{Jan Pieter\footnote{Initially unassigned}}
	{2}{2}

\subtotal{6}{18}

\subheading{
	Added during iteration\footnote{Significant tasks not planned before.}
}

\task{}{Ant build evironment}{Jan Pieter}{-}{2} 

\subtotal{-}{2}

\grandtotal{65}{??}
\end{tabularx}
\end{center}

\section{Reflection on this iteration}
In this section we will give a quick review on this iteration. First we will comment on our initial planning and second we will review the implementation phase and the workflow.
\subsection{Planning}
The estimated efforts were not that far off the actual time needed for implementation and we are confident that our estimations will improve as the project continues. We did, however, forget to plan a few obvious tasks. For example, code reviewing, tool selection (testing HTML responses and JSON-serialisation) and creating a building environment are tasks that could have been foreseen. Also the time needed for everyone to setup their working environment was underestimated.

A lot of these tasks were one-time efforts, but they can consume precious time. We think the number of these small tasks will reduce, as this was the very first iteration. We now have a starting point that can be built on. For next iterations, we really have to think things through, to make sure there will be as little unforeseen subtasks as possible.

\pagebreak
\subsection{Implementation}
The implementation phase for this iteration didn't go as smoothly as we hoped it would go, but it did gave us a few points to work on. One of these points is communication; there was little communication between members. This, combined with a low number of commits, made it hard for members to see what everyone was working on and what the overall progress was. Only when the deadline was approaching, the commits started to flow and the project advanced quickly.
Late commits make code reviewing and integration testing harder. That is why we want to commit early and commit often in the next iteration.\\

The client-side scaffolding didn't go as planned either. One of the issues was the QUnit testing. Implementing tests works well, but the feedback once something goes wrong is often rather useless. So the debugging took more time than expected. The Javascript scaffolding was not completed within the scheduled time because of a lack of communication (as more people could have helped). This has been pointed out and will be improved. In the next iteration, this point has priority and will be finished as soon as possible.\\

We are happy with the delivered code and product. Test-driven development was applied correctly and we think we have built a solid base. For this base we needed a few (previously undocumented) tools:

\paragraph{Build environment}
We decided to use Apache ant to build the sources and provide easy access to the JUnit test suites. Currently, four targets are present.

\paragraph{JSON serialisation}
Selected the library from \url{http://json.org/java} since it is lightweight.

\paragraph{HTML/JSON Unit testing}
We have selected HtmlUnit (\url{http://htmlunit.sourceforge.net/}) to help test the server applets. This unit acts as a browser (without the GUI) and fetches pages from the server. This helps testing the JSON responses. 
Unfortunately, we have found no easy way to compare JSON objects in Java. That is why we are comparing the objects as strings. JSON objects are unordered by definition, but we make sure the objects have a predictable layout (alphabetical order).

\end{document}
