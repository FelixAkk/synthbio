%
\documentclass[a4paper]{article}

% scrumplan-common.tex

\usepackage[utf8]{inputenc}
\usepackage{fullpage}
\usepackage{amsmath,amssymb}
\usepackage{tabularx,multirow}
\usepackage[colorlinks,linkcolor=blue]{hyperref} % use colored text in stead of ugly boxes
%~ \usepackage[toc]{multitoc} % Nice two-column TOC

\usepackage{ifthen}

% new line after paragraph title.
\makeatletter
\renewcommand\paragraph{\@startsection{paragraph}{4}{\z@}%
  {-3.25ex\@plus -1ex \@minus -.2ex}%
  {1.5ex \@plus .2ex}%
  {\normalfont\normalsize\bfseries}}
\makeatother


\setlength{\extrarowheight}{3pt}
\newcommand{\githubissue}[1]{\href{https://github.com/FelixAkk/synthbio/issues/#1}{\##1}}
\newcommand{\githubmilestone}[1]{\url{https://github.com/FelixAkk/synthbio/issues?milestone=#1}}
\newcommand{\thickhline}{\noalign{\hrule height 0.8pt}}

\newcommand{\tasktableheading}{
	\multicolumn{3}{c}{} & \multicolumn{2}{c}{Effort} \\
	& \textbf{Task} & Developer & estimated & actual \\\hline
}

\newcommand{\subheading}[1]{
 & \multicolumn{2}{l}{\bf #1} \\ \hline
}

\newcommand{\subtotal}[2]{
	\hline
	\multicolumn{3}{r}{\textbf{Total}}
	& #1
	& #2 \\[6mm]
}

\newcommand{\grandtotal}[2]{
	\thickhline
	\multicolumn{3}{r}{\textbf{Grand Total}}
		&
	#1 &
	#2 \\
}

%task command to 
\newcommand{\task}[5]{
	\ifthenelse{\equal{#1}{} \or \equal{#10}{0}}{-}{\githubissue{#1}} &
	#2 & % description
	#3 & % developer
	#4 & % estimated effort
	#5   % actual effort
	\\
}



\title{Project Zelula - SCRUM plan \#5}

\author{Group 5/E:\\
Felix Akkermans \\
Niels Doekemeijer \\
Thomas van Helden \\
Albert ten Napel \\
Jan Pieter Waagmeester}

\begin{document}

\maketitle

%~ • The selection of a set of features (important features first).
%~ • A List of the tasks for each feature.
%~ • The assignment of students to tasks.
%~ • An estimation of the effort per task.
%~ • The actual effort per task (after the iteration is done).
%~ • Short reflection on the main problems and adjustments of the iteration planning
\section{Introduction}
In this SCRUM plan we define the tasks we want to complete in this sprint. Every task is assigned to one or two developers and an estimate is provided for the effort required.

For each SCRUM run a milestone is created at GitHub, with issues for the tasks selected. The issues for this milestone can be found on: \githubmilestone{12}.

\section{Selection and assignment of tasks}
This is the fifth sprint of our project. 

\paragraph{Available time}
The time available is five mornings in two weeks by five people. That's about twenty hours per person. Including meetings we come to an estimate of 90 hours of actual working. 

A list of tasks, assignments and effort estimations is included in a table.

\begin{center}
\begin{tabularx}{\textwidth}{r p{7.5cm} | l | cc}
\tasktableheading

\task{81}
	{Server: Fix simulation results}
	{Albert}
	{8}{5}

\task{86}
	{Refactoring circuit servlet}
	{Jieter}
	{3}{4}

\task{87}
	{Server: Read new CSV input}
	{Jieter}
	{2}{4}

\task{88}
	{Client: Tooltip graph}
	{Niels}
	{2}{Actual}

\task{73}
	{Client: Resize}
	{Felix}
	{6}{Actual}

\task{66}
	{Client: Acceptance testing}
	{Felix + Thomas}
	{8}{7}

\task{71}
	{Client: Compound Gates}
	{Niels/Thomas}
	{12}{14}

\task{77}
	{Finish Scrum plan 5}
	{Felix}
	{2}{Actual}

\task{52}
	{Final report: Key Problems/Solutions}
	{Thomas + rest}
	{4}{2+}

\task{55}
	{Final report: Reflection Teamwork}
	{Everyone}
	{5 * 1}{Actual}
	
\task{89}
	{Final report: Introduction}
	{Jieter}
	{1}{Actual}
	
\task{90}
	{Final report: Description of product}
	{Jieter}
	{4}{Actual}
	
\task{92}
	{Final report: Design and implementation process}
	{First to be finished}
	{5}{Actual}

\task{62}
	{Confirmation on close}
	{}
	{0}{dropped}

\task{}
	{Code Review}
	{Everyone}
	{5 * 3}{6+2}

\subtotal{77}{-}
 
\subheading{
	Optional tasks
}

\task{43}
	{Client: Polish circuit styles}
	{}
	{}{Actual}

\task{60}
	{Automated QUnit tests from ant}
	{Albert}
	{}{1}

\task{14}
	{Client: Compile .less server side}
	{}
	{}{dropped}

\task{13}
	{Client:Assign keyboard shortcuts}
	{Felix}
	{}{Actual}

\task{84}
	{Client: Input and output fields resize and display open connection}
	{}
	{}{Actual}

\task{82}
	{Client: Migrate stuff to simulation tab}
	{}
	{}{Actual}

\task{65}
	{Save input/output gate position}
	{}
	{}{Actual}

\task{43}
	{Client: Deleting of wires and gates using delete}
	{}
	{}{Actual}

\task{43}
	{Client: Highlighting/selecting gates to move multiple or delete multiple}
	{}
	{}{dropped}

\task{43}
	{Client: Input and output fields resize and display open connections}
	{}
	{}{Actual}
	
\task{91}
	{Client: Table with simulation results}
	{}
	{}{dropped}



\subtotal{-}{-}
\subheading{Extra tasks}

\task{}
	{Fixing the simulator: writing our own simulator}
	{Jan Pieter}
	{-}{20}
	
\task{}
	{Exporting graph: serve our own SVG rasterizer}
	{Jan Pieter}
	{-}{4}
	
\subtotal{-}{-}

\grandtotal{-}{-}
\end{tabularx}
\end{center}

\section{Reflection on this iteration}

\paragraph{Solver issues}
During this iteration we discovered a problem with the connection to the solver which was implemented. It showed correct results for the transcription and translation reactions, but degradation did not work. After discussing things with Alexey, Jan Pieter decided to write a solver on its own, which initially took about 12 hours of work. Cleaning up the code and some other adjustments took another 8 hours.

During the development of our own solver, Albert worked together with Alexey to fix the problems in the first approach. The result of this is the availability of two solvers in our GUI.

\paragraph{Graph rasterisation, servlet issues}
The graph library used in our GUI provides a way to export the graph to different image formats. This exporting requires a special server which is provided by the author of the library, but we wanted to be independent of that, so we decided to provide the rasterisation service by our own server code.

At first, it seemed very straight-forward, but quite some time was absorbed by the confusion about Tomcat's class-path. It turns out Tomcat only looks in \verb|WEB-INF/lib/*| and not in it's subdirectories, resulting in a huge stack of \verb|NoClassDefFoundErrors| to fix.


\end{document}
