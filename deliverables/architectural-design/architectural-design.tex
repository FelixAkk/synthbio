\documentclass[a4paper]{article}
\usepackage[utf8]{inputenc}
\usepackage{fullpage}
\usepackage{amsmath,amssymb}
\usepackage[colorlinks]{hyperref} % use colored text in stead of ugly boxes
\usepackage[toc]{multitoc} % Nice two-column TOC

\usepackage{pgf}
\usepackage{tikz}
\usepackage{pictures/tikz-uml}

\title{Programming Life - Architectural Design }

\author{Group 5/E:\\
Felix Akkermans \\
Niels Doekemeijer \\
Thomas van Helden \\
Albert ten Napel \\
Jan Pieter Waagmeester}

\begin{document}
\maketitle

\vfill

\small{\tableofcontents}
\pagebreak
\section{Introduction}
\subsection{Purpose of the System}
\subsection{Design Goals}
\subsection{Definitions, acronyms and abbreviations}
\subsection{References}
\subsection{Overview}

\section{Proposed software architecture}
\subsection{Overview}
\subsection{Subsystem Decomposition}
(which sub-systems and dependencies are there between the sub-systems?)

\subsubsection{Interface (API) of each sub-system}
\begin{figure}[h!]
\caption{Examples of JSON responses}
Generic response:
\begin{verbatim}
{
  "success": boolean,
  "message": string,
  "data": object
}
\end{verbatim}
Response of getFiles(): \begin{verbatim}
{
  "success": true,
  "message": "",
  "data": [
    "example1.syn",
    "d-flipflop.syn",
    "xor-gate.syn"
  ]
}
\end{verbatim}
Response for getFile(fileName) for a non-existing file:
\begin{verbatim}

{
  "success": false,
  "message": "Requested file not found.",
  "data": {}
}
\end{verbatim}
\end{figure}
\subsection{Hardware/Software Mapping}
Our architecture is composed of two major elements, the client implemented as a web application and the server, implemented in Java. The browser has to be a recent version of one of the major browser\footnote{We'll test in Google Chrome and Firefox}, the supporting system is not relevant. The server can be run on the same machine, which is likely to be the same machine during the development stage.

\subsection{Persistent Data Management}
(file/ database, database design)
\subsection{Global Resource Handling and Access Control for the different actors}
\subsection{Concurrency}
(which processes run in parallel, how do they communicate, how are deadlocks prevented?)
\subsection{Boundary Conditions}
(how is the system started and stopped, what happens
in case of a system crash)


\end{document}
