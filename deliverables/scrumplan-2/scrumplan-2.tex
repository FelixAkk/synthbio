\documentclass[a4paper]{article}

% scrumplan-common.tex

\usepackage[utf8]{inputenc}
\usepackage{fullpage}
\usepackage{amsmath,amssymb}
\usepackage{tabularx,multirow}
\usepackage[colorlinks,linkcolor=blue]{hyperref} % use colored text in stead of ugly boxes
%~ \usepackage[toc]{multitoc} % Nice two-column TOC

\usepackage{ifthen}

% new line after paragraph title.
\makeatletter
\renewcommand\paragraph{\@startsection{paragraph}{4}{\z@}%
  {-3.25ex\@plus -1ex \@minus -.2ex}%
  {1.5ex \@plus .2ex}%
  {\normalfont\normalsize\bfseries}}
\makeatother


\setlength{\extrarowheight}{3pt}
\newcommand{\githubissue}[1]{\href{https://github.com/FelixAkk/synthbio/issues/#1}{\##1}}
\newcommand{\githubmilestone}[1]{\url{https://github.com/FelixAkk/synthbio/issues?milestone=#1}}
\newcommand{\thickhline}{\noalign{\hrule height 0.8pt}}

\newcommand{\tasktableheading}{
	\multicolumn{3}{c}{} & \multicolumn{2}{c}{Effort} \\
	& \textbf{Task} & Developer & estimated & actual \\\hline
}

\newcommand{\subheading}[1]{
 & \multicolumn{2}{l}{\bf #1} \\ \hline
}

\newcommand{\subtotal}[2]{
	\hline
	\multicolumn{3}{r}{\textbf{Total}}
	& #1
	& #2 \\[6mm]
}

\newcommand{\grandtotal}[2]{
	\thickhline
	\multicolumn{3}{r}{\textbf{Grand Total}}
		&
	#1 &
	#2 \\
}

%task command to 
\newcommand{\task}[5]{
	\ifthenelse{\equal{#1}{} \or \equal{#10}{0}}{-}{\githubissue{#1}} &
	#2 & % description
	#3 & % developer
	#4 & % estimated effort
	#5   % actual effort
	\\
}



\title{Project Zelula - SCRUM plan \#2}

\author{Group 5/E:\\
Felix Akkermans \\
Niels Doekemeijer \\
Thomas van Helden \\
Albert ten Napel \\
Jan Pieter Waagmeester}

\begin{document}
\maketitle

%~ • The selection of a set of features (important features first).
%~ • A List of the tasks for each feature.
%~ • The assignment of students to tasks.
%~ • An estimation of the effort per task.
%~ • The actual effort per task (after the iteration is done).
%~ • Short reflection on the main problems and adjustments of the iteration planning
\section{Introduction}
In this SCRUM plan we define the tasks we want to complete in this sprint. Every task is assigned to one or two developers and an estimate is provided for the effort required.

For each SCRUM run a milestone is created at GitHub, with issues for the tasks selected. The issues for this milestone can be found on: \githubmilestone{9}.

\section{Selection and assignment of tasks}
Because this is the second iteration, we think we will find less unexpected tasks. The available time is more than last time, so we want to do more work than in the last iteration. For this iteration, the goal is to end up with a GUI that supports basic circuit design and a back-end that supports basic circuit handling (loading/saving/validating).

\paragraph{Available time}
The time available is five mornings in two weeks by five people. That's about twenty hours per person, however, some time should be reserved for the meetings. So the total effort available is about 90 man hours.

A list of tasks, assignments and effort estimations is included in a table.


\paragraph{Presentation}
Results for this iteration will be presented Friday May 4, 12:15 in DW-PC 0.010.


\begin{center}
\begin{tabularx}{\textwidth}{r p{8cm} | l | cc}
\tasktableheading

\task{18}
	{Server: Save circuit}
	{Jan Pieter}
	{\(\downarrow\)\footnote{Included in time below}}{\(\downarrow\)}

\task{19}
	{Server: Load circuit}
	{Jan Pieter}
	{\(\downarrow\)}{\(\downarrow\)}

\task{20}
	{Server: Validate circuit}
	{Jan Pieter}
	{12}{Actual}

\task{26}
	{Server: Serve saved circuits}
	{Jan Pieter}
	{3}{Actual}

\task{21}
	{Server: Connection to simulator}
	{Albert}
	{12}{Actual}

\task{27}
	{Server: Convert circuit to simulator input}
	{Albert}
	{3}{Actual}

\task{4}
	{Client: Javascript scaffolding}
	{Thomas}
	{10}{Actual}

\task{22}
	{Client: Gate scaffolding and rendering}
	{Felix}
	{\(\downarrow\)}{\(\downarrow\)}

\task{24}
	{Client: Drag-and-drop gates to working area}
	{Felix}
	{\(\downarrow\)}{\(\downarrow\)}

\task{25}
	{Client: Move gates in the working area}
	{Felix}
	{14}{Actual}

\task{30}
	{Client: Draw wires between gates}
	{Niels}
	{\(\downarrow\)}{\(\downarrow\)}

\task{31}
	{Client: Draw input/output wires}
	{Niels}
	{14}{Actual}

\task{28}
	{Reflection scrum plan 2}
	{Thomas}
	{3}{Actual}

\task{29}
	{Scaffolding scrum plan 3}
	{Thomas}
	{1}{Actual}

\task{32}
	{Code review}
	{Everyone}
	{5 * 4}{Actual}

\subtotal{92}{-}
 
\subheading{
	Optional tasks\footnote{Things from next iterations that could be done if sufficient time is available}
}

\task{0}
	{Server: Simulate circuit}
	{-}
	{Planned}{Actual}

\task{0}
	{Client: Validate circuit}
	{-}
	{Planned}{Actual}

\task{0}
	{Client: Load circuit}
	{-}
	{Planned}{Actual}

\task{0}
	{Client: Save circuit}
	{-}
	{Planned}{Actual}

\task{0}
	{Client: Specify proteins for wires}
	{-}
	{Planned}{Actual}

\subtotal{-}{-}

\grandtotal{92}{-}
\end{tabularx}
\end{center}

\section{Reflection on this iteration}
In this section we will give a quick review on this iteration. \\

Extra (previously undocumented) tools:

\paragraph{jsPlumb}
We have chosen jsPlumb (\url{http://jsplumb.org/}) as framework for the (visual) connections in the GUI. This framework makes it easy to define anchors and draw wires between these anchors. \\
Unfortunately, the framework did not match up to all our wishes. Small changes to the library were necessary to allow jsPlumb to automatically create anchors from which multiple wires can be drawn (these kind of anchors are used in the connector from which input signals originate).

\end{document}
