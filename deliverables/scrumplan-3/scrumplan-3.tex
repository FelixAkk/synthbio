\documentclass[a4paper]{article}

% scrumplan-common.tex

\usepackage[utf8]{inputenc}
\usepackage{fullpage}
\usepackage{amsmath,amssymb}
\usepackage{tabularx,multirow}
\usepackage[colorlinks,linkcolor=blue]{hyperref} % use colored text in stead of ugly boxes
%~ \usepackage[toc]{multitoc} % Nice two-column TOC

\usepackage{ifthen}

% new line after paragraph title.
\makeatletter
\renewcommand\paragraph{\@startsection{paragraph}{4}{\z@}%
  {-3.25ex\@plus -1ex \@minus -.2ex}%
  {1.5ex \@plus .2ex}%
  {\normalfont\normalsize\bfseries}}
\makeatother


\setlength{\extrarowheight}{3pt}
\newcommand{\githubissue}[1]{\href{https://github.com/FelixAkk/synthbio/issues/#1}{\##1}}
\newcommand{\githubmilestone}[1]{\url{https://github.com/FelixAkk/synthbio/issues?milestone=#1}}
\newcommand{\thickhline}{\noalign{\hrule height 0.8pt}}

\newcommand{\tasktableheading}{
	\multicolumn{3}{c}{} & \multicolumn{2}{c}{Effort} \\
	& \textbf{Task} & Developer & estimated & actual \\\hline
}

\newcommand{\subheading}[1]{
 & \multicolumn{2}{l}{\bf #1} \\ \hline
}

\newcommand{\subtotal}[2]{
	\hline
	\multicolumn{3}{r}{\textbf{Total}}
	& #1
	& #2 \\[6mm]
}

\newcommand{\grandtotal}[2]{
	\thickhline
	\multicolumn{3}{r}{\textbf{Grand Total}}
		&
	#1 &
	#2 \\
}

%task command to 
\newcommand{\task}[5]{
	\ifthenelse{\equal{#1}{} \or \equal{#10}{0}}{-}{\githubissue{#1}} &
	#2 & % description
	#3 & % developer
	#4 & % estimated effort
	#5   % actual effort
	\\
}



\title{Project Zelula - SCRUM plan \#3}

\author{Group 5/E:\\
Felix Akkermans \\
Niels Doekemeijer \\
Thomas van Helden \\
Albert ten Napel \\
Jan Pieter Waagmeester}

\begin{document}
\maketitle

%~ • The selection of a set of features (important features first).
%~ • A List of the tasks for each feature.
%~ • The assignment of students to tasks.
%~ • An estimation of the effort per task.
%~ • The actual effort per task (after the iteration is done).
%~ • Short reflection on the main problems and adjustments of the iteration planning
\section{Introduction}
In this SCRUM plan we define the tasks we want to complete in this sprint. Every task is assigned to one or two developers and an estimate is provided for the effort required.

For each SCRUM run a milestone is created at GitHub, with issues for the tasks selected. The issues for this milestone can be found on: \githubmilestone{10}.

\section{Selection and assignment of tasks}
This is the third sprint of our project. During this sprint we will implement validation, evaluation and simulation of a circuit. We will also look at saving and loading a circuit. If all goes according to plan we can start reusing previously build circuits and simulate the outcome of a circuit. If we do this it means we have the product we planned implementing at the beginning of this project. Afterwards it will only be fine tuning.

\paragraph{Available time}
The time available is four mornings in two weeks by five people. That's about sixteen hours per person.Including meetings we come to an estimate of 70 hours of actual working. However, this is below the amount of hours spend with respect to ECTS, so we will plan spend a little more.

A list of tasks, assignments and effort estimations is included in a table.

\begin{center}
\begin{tabularx}{\textwidth}{r p{8cm} | l | cc}
\tasktableheading


\task{}
	{Final report: Key Problems/Solutions}
	{}
	{Estimate}{Actual}

\task{}
	{Client: Specify protein for wires}
	{}
	{Estimate}{Actual}

\task{}
	{Client: Specify Input}
	{}
	{Estimate}{Actual}

\task{}
	{Client: Simulate Circuit}
	{}
	{Estimate}{Actual}

\task{}
	{Server: Simulate Circuit}
	{}
	{Estimate}{Actual}

\task{}
	{Wiki: Simulator}
	{}
	{Estimate}{Actual}

\task{}
	{Client: Save Circuit}
	{}
	{Estimate}{Actual}

\task{}
	{Server: Convert circuit to simulator input}
	{}
	{Estimate}{Actual}

\task{}
	{Client: Load Circuit}
	{}
	{Estimate}{Actual}

\task{}
	{Client: Circuit bookkeeping}
	{Felix}
	{Estimate}{Actual}

\task{}
	{Code Review}
	{}
	{Estimate}{Actual}

\task{}
	{Final Scrumplan-3}
	{}
	{Estimate}{Actual}

\task{}
	{Scaffolding Scrumplan-4}
	{}
	{Estimate}{Actual}

\subtotal{}{}
 
\subheading{
	Optional tasks\footnote{Things from next iterations that could be done if sufficient time is available}
}

\task{}
	{Polish model/grid}
	{-}
	{Estimate}{Actual}

\task{}
	{Proper gate-mouse location}
	{-}
	{Estimate}{Actual}

\task{}
	{Input Output visualisation}
	{}
	{Estimate}{Actual}

\task{}
	{Protein selection for wires drop-down menu}
	{-}
	{Estimate}{Actual}

\task{}
	{Final report: Reflection on teamwork}
	{}
	{Estimate}{Actual}



\subtotal{-}{-}

\grandtotal{}{-}
\end{tabularx}
\end{center}

\section{Reflection on this iteration}
In this section we will give a quick review on this iteration. \\


\end{document}
