%
\documentclass[a4paper]{article}

% scrumplan-common.tex

\usepackage[utf8]{inputenc}
\usepackage{fullpage}
\usepackage{amsmath,amssymb}
\usepackage{tabularx,multirow}
\usepackage[colorlinks,linkcolor=blue]{hyperref} % use colored text in stead of ugly boxes
%~ \usepackage[toc]{multitoc} % Nice two-column TOC

\usepackage{ifthen}

% new line after paragraph title.
\makeatletter
\renewcommand\paragraph{\@startsection{paragraph}{4}{\z@}%
  {-3.25ex\@plus -1ex \@minus -.2ex}%
  {1.5ex \@plus .2ex}%
  {\normalfont\normalsize\bfseries}}
\makeatother


\setlength{\extrarowheight}{3pt}
\newcommand{\githubissue}[1]{\href{https://github.com/FelixAkk/synthbio/issues/#1}{\##1}}
\newcommand{\githubmilestone}[1]{\url{https://github.com/FelixAkk/synthbio/issues?milestone=#1}}
\newcommand{\thickhline}{\noalign{\hrule height 0.8pt}}

\newcommand{\tasktableheading}{
	\multicolumn{3}{c}{} & \multicolumn{2}{c}{Effort} \\
	& \textbf{Task} & Developer & estimated & actual \\\hline
}

\newcommand{\subheading}[1]{
 & \multicolumn{2}{l}{\bf #1} \\ \hline
}

\newcommand{\subtotal}[2]{
	\hline
	\multicolumn{3}{r}{\textbf{Total}}
	& #1
	& #2 \\[6mm]
}

\newcommand{\grandtotal}[2]{
	\thickhline
	\multicolumn{3}{r}{\textbf{Grand Total}}
		&
	#1 &
	#2 \\
}

%task command to 
\newcommand{\task}[5]{
	\ifthenelse{\equal{#1}{} \or \equal{#10}{0}}{-}{\githubissue{#1}} &
	#2 & % description
	#3 & % developer
	#4 & % estimated effort
	#5   % actual effort
	\\
}



\title{Project Zelula - SCRUM plan \#3}

\author{Group 5/E:\\
Felix Akkermans \\
Niels Doekemeijer \\
Thomas van Helden \\
Albert ten Napel \\
Jan Pieter Waagmeester}

\begin{document}

\maketitle

%~ • The selection of a set of features (important features first).
%~ • A List of the tasks for each feature.
%~ • The assignment of students to tasks.
%~ • An estimation of the effort per task.
%~ • The actual effort per task (after the iteration is done).
%~ • Short reflection on the main problems and adjustments of the iteration planning
\section{Introduction}
In this SCRUM plan we define the tasks we want to complete in this sprint. Every task is assigned to one or two developers and an estimate is provided for the effort required.

For each SCRUM run a milestone is created at GitHub, with issues for the tasks selected. The issues for this milestone can be found on: \githubmilestone{10}.

\section{Selection and assignment of tasks}
This is the third sprint of our project. During this sprint we will implement validation, evaluation and simulation of a circuit. We will also look at saving and loading a circuit. If all goes according to plan we can start reusing previously build circuits and simulate the outcome of a circuit. If we do this it means we have the product we planned implementing at the beginning of this project. Afterwards it will only be fine tuning.

\paragraph{Available time}
The time available is four mornings in two weeks by five people. That's about sixteen hours per person.Including meetings we come to an estimate of 70 hours of actual working. However, this is below the amount of hours spend with respect to ECTS, so we will plan spend a little more.

A list of tasks, assignments and effort estimations is included in a table.

\begin{center}
\begin{tabularx}{\textwidth}{r p{8cm} | l | cc}
\tasktableheading

\task{42}
	{Client: Specify protein for wires by drop-down menu}
	{Thomas}
	{3}{12}

\task{44}
	{Client: Specify input signals}
	{Jan-Pieter}
	{16}{20}[2mm]


\task{22}
	{Client: Circuit bookkeeping}
	{Felix+Niels}
	{\multirow{3}{*}{$\Bigg\}$ 2*15}}{5+15}
		
\task{45}
	{Client: Save Circuit}
	{Felix+Niels}
	{}{5+0}

\task{46}
	{Client: Load Circuit}
	{Felix+Niels}
	{}{8+4}[2mm]
	
\task{47}
	{Server: Simulate Circuit}
	{Albert}
	{2}{2}
		
\task{48}
	{Server: Convert circuit to simulator input}
	{Albert}
	{14}{16}

\task{49}
	{Wiki: Simulator}
	{Albert}
	{1}{0}[2mm]

\task{}
	{Code Review}
	{All}
	{5*4}{2+3+4+1+1} % jieter did 4 hours

\task{50}
	{Final Scrumplan-3}
	{Thomas}
	{2}{3}

\task{51}
	{Scaffolding Scrumplan-4}
	{Thomas}
	{1}{1}

\task{52}
	{Final report: Key Problems/Solutions}
	{Thomas}
	{4}{0}

\subtotal{93}{92}
 
\subheading{
	Optional tasks\footnote{Things from next iterations that could be done if sufficient time is available}
}

\task{43}
	{Polishing modelling/grid styles and workflow}
	{Felix+Niels}
	{-}{1+1}

\task{53}
	{Client: Output visualisation}
	{-}
	{-}{-}

\task{54}
	{Client: Simulate Circuit}
	{-}
	{-}{-}

\task{55}
	{Final report: Reflection on teamwork}
	{-}
	{-}{-}


\subtotal{-}{2}

\subheading{
	Added during iteration
}

\task{}
	{JSLint ant target and style review.\footnote{To ensure some best practices in our JavaScript code, including coding style, we use JSLint/JSHint}}
	{Jan Pieter}
	{-}{2}
		
\task{}
	{Client: validate Circuit\footnote{For some reason, this task was not in the initial scrum plan \#3, but it was an optional task in \#2.}}
	{Jan Pieter}
	{-}{2}

\subtotal{-}{4}

\grandtotal{93}{98}
\end{tabularx}
\end{center}

\pagebreak

\section{Reflection on this iteration}
During this iteration one of the main problems we encountered were in late commits and a lack of communication when someone was getting behind on schedule. This would lead to a lack of time to correct late work or to help someone catch up with the schedule. During the scrum-reflection we discussed this and noticed that as before this is still a problem. Some improvements were noticed this scrum iteration, and we further emphasized that this is still to be worked on and requires more attention during in our group-process. Everyone agreed on this and plans to show improvements in this aspect.

Extending on this issue we also noticed that at the end of the scrum iteration there were still quite some issues open that should have been closed. This requires more attention so the state of the project issues on GitHub more accurately reflects it's current state in the code.

Another thing we noticed is that our actual time expenditure on code review was below of that what was planned. We adjusted our expectations for the next scrum iterations so this would be more realistic.\\

\subsection{Planning}
We had planned to quite some work this iteration even though there was a holiday during this scrum run. We were requested to have a working program during the next sprint, so we looked at what was still missing and aimed to complete a great proportion of that during this sprint. We mainly shoved the Circuit Simulation on the client side to the next scrum run. The rest of the big chunks we tried to tackle this iteration. We also had a little bit of work left from the previous sprint so we planned that in as well. We acknowledged that our final report will be a big piece of work as well, so we decided to work a little bit in advance on that.
 
\subsection{Implementation}
During the implementation of all of our functions we encountered some difficulties. We had some difficulties implementing dropdown menus on wires. We didn't expect this but it seems jsPlumb combined with bootstrap gave quite some errors. We didn't manage to fully complete this part so we pushed the remainder on to the next sprint.

Other than this we mainly made a lot of small improvements to our program. We did not encounter great problems, instead we found a lot of room for minor improvements. We took these enhancements and set them as optional in our next sprint.



\end{document}
